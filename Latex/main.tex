\documentclass[11pt,a4paper]{article}

% 1. Codificación e Idioma
\usepackage[utf8]{inputenc}
\usepackage[spanish, es-tabla]{babel}

% 2. Paquete de gestión de captiones (añadir esto suele fijar el error)
\usepackage{caption} 

% 3. Formato y Autores
\usepackage[margin=2.5cm]{geometry}
\usepackage{authblk} 

% 4. Bibliografía APA (Cargar al final del preámbulo)
\usepackage[style=apa, backend=biber]{biblatex}
\addbibresource{referencias.bib}

\title{\textbf{Título del Artículo Científico}}
\author{Nombre del Autor}
\affil{Máster TECI - Series Temporales}
\date{}

\begin{document}
	\maketitle
	
	% --- Resumen ---
	\begin{abstract}
		\noindent El resumen debe tener un máximo de 250 palabras. Aquí se debe describir de forma concisa el objetivo del estudio, la metodología aplicada, los resultados principales y las conclusiones alcanzadas. Se empleará un lenguaje científico y preciso durante toda la redacción.
	\end{abstract}
	
	\noindent \textbf{Palabras clave:} Serie temporal, Predicción, Máster TECI, [Palabra 4].
	
	
	% Sustituye el \hr por esto si quieres una línea:
	\vspace{0.5cm}
	\hrule
	\vspace{0.5cm}
	\section{Introducción}
	La introducción debe contextualizar el problema de las series temporales tratado. Es fundamental incluir citas bibliográficas que apoyen el estado del arte siguiendo el formato APA \parencite{ejemplo2026}.
	
	\section{Metodología}
	Descripción detallada del modelo realizado, los datos utilizados y los procedimientos técnicos aplicados.
	
	\section{Resultados}
	Presentación de los hallazgos obtenidos. Se pueden incluir tablas y figuras para mayor claridad.
	
	\section{Discusión}
	Análisis de los resultados en comparación con la literatura consultada. Debe incluirse la \textbf{autoevaluación del modelo realizado}, analizando sus fortalezas y limitaciones.
	
	\section{Conclusiones}
	Resumen de las aportaciones principales del trabajo y posibles líneas futuras.
	
	% --- Bibliografía ---
	\newpage
	\printbibliography[title={Referencias}]
	
\end{document}